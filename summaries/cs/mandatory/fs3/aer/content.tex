%% Bluespec SystemVerilog language definition for the LaTeX listings package
%% Copyright 2013 Yury Bayda
\lstdefinelanguage[Bluespec]{Verilog}{
	sensitive = true,
	morekeywords = [1]{
		%% SystemVerilog keywords
		begin,
		bit,
		break,
		case,
		continue,
		default,
		do,
		else,
		end,
		endcase,
		function,
		endfunction,
		endinterface,
		endmodule,
		endpackage,
		enum,
		export,
		extern,
		for,
		if,
		import,
		int,
		interface,
		local,
		localparam,
		matches,
		module,
		package,
		tagged,
		type,
		typedef,
		struct,
		union,
		void,
		%% Bluespec keywords
		action,
		endaction,
		actionvalue,
		endactionvalue,
		ancestor,
		deriving,
		instance,
		endinstance,
		let,
		return,
		match,
		method,
		endmethod,
		par,
		endpar,
		powered_by,
		provisos,
		rule,
		endrule,
		rules,
		endrules,
		seq,
		endseq,
		schedule,
		typeclass,
		endtypeclass,
		clock,
		reset,
		noreset,
		no_reset,
		clocked_by,
		reset_by,
		default_clock,
		default_reset,
		output_clock,
		output_reset,
		input_clock,
		input_reset,
		same_family,
	},
	morekeywords = [2]{
		%% Frequently used predefined types
		Action,
		ActionValue,
		Integer,
		Nat,
		Bit,
		UInt,
		Int,
		Bool,
		Maybe,
		String,
		Either,
		Rules,
		Module,
		Clock,
		Reset,
		Power,
		TAdd, TSub, TMul, TDiv, TLog, TExp, TMax, TMin, SizeOf,
		%%
		Empty,
		Reg,
		RWire, Wire, BypassWire, PulseWire,
		RegFile,
		Vector,
		FIFO, FIFOF,
		%%
		Bits, Eq, Ord, Bounded,
		Arith, Literal, Bitwise, BitReduction, BitExtend,
		IsModule, TieOff, ToPut, ToGet,
		Add, Max, Log, FShow, Randomizable, RealLiteral,
		%% Attributes
		synthesize, noinline, doc, options,
		always_ready, always_enabled,
		ready, enable, result, prefix,
		fire_when_enabled, no_implicit_conditions,
		bit_blast, scan_insert,
		descending_urgency, preempts, conservative_implicit_conditions,
		internal_scheduling, method_scheduling,
		CLK, RST_N, RSTN, ungated_clock
	}, 
	morekeywords = [3]{
		%% Frequently used expressions
		True,
		False,
		mkReg, mkRegU, mkRegA, mkRWire, mkWire, mkFIFO, mkSizedFIFOUG, mkFIFO1,
		mkBypassWire, mkDWire, mkPulseWire,
		pack, unpack, zeroExtend, signExtend, truncate, extend,
		fromInteger, inLiteralRange, negate,
		valueOf, valueof, SizeOf, defaultValue,
		minBound, maxBound,
		signedShiftRight, div, mod, exp, log2, add, abs, max, min, quot, rem,
		fromMaybe, isValid, validValue, noAction,
		error, warning, message, messageM,
		nosplit, emptyRules, addRules, rJoin, rJoinPreempts, rJoinDescendingUrgency,
		fshow, newVector,
		%% System tasks
		\$display,
		\$displayb,
		\$displayh,
		\$displayo,
		\$write,
		\$writeb,
		\$writeh,
		\$writeo,
		\$fopen,
		\$fclose,
		\$fgetc,
		\$ungetc,
		\$fflush,
		\$fdisplay,
		\$fdisplayb,
		\$fdisplayh,
		\$fdisplayo,
		\$fwrite,
		\$fwriteb,
		\$fwriteh,
		\$fwriteo,
		\$stop,
		\$finish,
		\$dumpon,
		\$dumpoff,
		\$dumpvars,
		\$dumpfile,
		\$dumpflush,
		\$time,
		\$stime,
		\$signed,
		\$unsigned,
		\$test\$plusargs,
		\$format
	},
	morecomment = [l]{//},
	morecomment = [s]{/*}{*/},
	morestring = [b]",
	morestring = [b]'
}



\chapter{Bluespec}
	\section{Grundlagen}
		\subsection{Zerlegungshierarchie}
			\begin{itemize}
				\item Bluespec zerlegt ein großes Programm in kleinere Module
				\item Blätter der Hierarchie sind primitive Zustandselemente
				\item Auch Register sind Module!
				\item Module stellen Schnittstellen durch Methoden bereit
				\item Module enthalten Regeln, welche Methoden \textit{anderer} Module aufrufen
				\item Methoden können auch Methoden \textit{anderer} Module aufrufen
			\end{itemize}
		% end
		
		\subsection{Methodenarten}
			Bluespec kennt die folgenden Methodenarten:
			\paragraph{Value}
				\begin{itemize}
					\item Können den Zustand der Schaltung nicht verändern
					\item Können lokale Zwischenwerte berechnen
					\item Geben Daten zurück
				\end{itemize}
			% end
			
			\paragraph{Action}
				\begin{itemize}
					\item Können den Zustand der Schaltung ändern
					\item Können keine Daten zurück geben
				\end{itemize}
			% end
			
			\paragraph{Action-Value}
				\begin{itemize}
					\item Können den Zustand der Schaltung ändern
					\item Geben Daten zurück
				\end{itemize}
			% end
		% end
		
		\subsection{Zuweisungsoperatoren}
			Bluespec kennt die folgenden Zuweisungsoperatoren:
			\begin{description}
				\item[\texttt{=}] Legt den Wert einer Variable fest auf den Rückgabewert einer Methode/Funktion/\dots. (Der linke Teil wird dem rechten gleichgesetzt.)
				\item[\texttt{<-}] Führt eine Action-Value-Methode aus und setzt die Variable auf den Rückgabewert. Außerdem muss dieser Operator genutzt werden, wenn Instanzen eines Moduls erstellt werden. (Der rechte Teil wird ausgeführt und danach dem linken gleichgesetzt.)
				\item[\texttt{<=}] Verändert den Wert eines Registers (impliziter Aufruf von \texttt{\_write}).
			\end{description}
		% end
	% end
	
	\section{Ausführungssemantik}
		\subsection{\texttt{WILL\_FIRE}}
			Das \texttt{WILL\_FIRE} einer Methode setzt sich aus dem Guard der Methode/Regel und den Guards der aufgerufenen Methoden.
		% end
		
		\subsection{Parallelität}
			Die Aktionen innerhalb einer Regel werden Echtparallel ausgeführt und \textit{nach} der Auswertung der Regel den Registern zugewiesen. Zum Zeitpunkt des Lesens von Registern wird immer der alte Wert zurück gegeben.
			
			Hierdurch sind die folgenden Codebeispiele identisch und tauschen die Werte der Register \texttt{x} und \texttt{y}:
			\begin{figure}[ht]
				\centering
				\begin{subfigure}{0.4\textwidth}
					\begin{lstlisting}
rule swap;
	x <= y;
	y <= x;
endrule
					\end{lstlisting}
					\caption{Vertausche \texttt{x} und \texttt{y}}
				\end{subfigure}
				\begin{subfigure}{0.4\textwidth}
					\begin{lstlisting}
rule swap;
	y <= x;
	x <= y;
endrule
					\end{lstlisting}
					\caption{Vertausche \texttt{y} und \texttt{x}}
				\end{subfigure}
			\end{figure}
		% end
		
		\subsection{Nebenläufigkeit}
			\subsubsection{Ablaufplan}
				\paragraph{Ausführungsreihenfolge}
					Es werden möglichst viele Regeln \textit{innerhalb eines Taktes}, aber \textit{nicht zeitgleich} ausgeführt, solange dies nicht zu Konflikten führt.
				% end
				
				\paragraph{Beeinflussung}
					Der Ablaufplan kann mit einigen Attributen beeinflusst werden, siehe \ref{sec:bsvkomponenten}.
				% end
			% end
			
			\subsubsection{Konflikte}
				Bei der der Erstellung des Ablaufplans kann es zu Konflikten kommen, da es (bei manchem Methoden) Einschränkungen gibt, in welcher Reihenfolge diese ausgeführt werden:
				\begin{table}[ht]
					\centering
					\begin{tabular}{c c}
						Einschränkung & Bedeutung: Regeln mit Aufrufern von \texttt{mA} und \texttt{mB} können \dots \\
						\hline
						\texttt{mA} \textit{konfliktfrei} \texttt{mB} & \dots nebenläufig feuern (beliebige Reihenfolge). \\
						\texttt{mA < mB} & \dots nebenläufig feuern, solange \texttt{mA} vor \texttt{mB} ausgeführt wird. \\
						\texttt{mB < mA} & \dots nebenläufig feuern, solange \texttt{mB} vor \texttt{mA} ausgeführt wird. \\
						\texttt{mA} \textit{konflikt} \texttt{mB} & \dots nicht nebenläufig feuern. \\
					\end{tabular}
				\end{table}
				
				\paragraph{Häufige Ursachen}
					\begin{description}[leftmargin = 5cm]
						\item[Rule Ordering Conflict] Zustandselemente können nur einmal je Takt umschalten
							\begin{itemize}
								\item Lesen eines geänderten Zustandes im selben Takt ist nicht (ohne weiteres) möglich
							\end{itemize}
						\item[Rule Resource Conflict] Hardware-Ressourcen (z.B. Drähte) können nur einmal je Takt genutzt werden
					\end{description}
				% end
			% end
		% end
		
		\subsection{Pipelines}
			\paragraph{Dynamische Pipeline}
				\begin{itemize}
					\item Latenz ist nur datenabhängig variabel
					\item Gegenteil: statische Pipeline
				\end{itemize}
			% end
			
			\paragraph{Elastische Pipeline}
				\begin{itemize}
					\item Daten bewegen sich mit unterschiedlichem Fortschritt durch die Pipeline
					\item Gegenteil: inelastische/starre Pipeline
				\end{itemize}
			% end
		% end
	% end
	
	\section{Synthetisierung}
		Die Synthetisierung kann mit einigen Attributen beeinflusst werden, siehe \ref{sec:bsvkomponenten}.
	% end
	
	\section{Komponenten}
		\label{sec:bsvkomponenten}

		\subsection{FIFO}
			\paragraph{Normale FIFO}
				\begin{itemize}
					\item Ist die FIFO leer, ist kein \texttt{deq} möglich (auch wenn zeitgleich ein \texttt{enq} stattfindet).
					\item Ist die FIFO voll, ist kein \texttt{enq} möglich (auch wenn zeitgleich ein \texttt{deq} stattfindet).
				\end{itemize}
			% end
			
			\paragraph{Pipeline FIFO}
				\begin{itemize}
					\item Ist die FIFO leer, ist kein \texttt{deq} möglich.
					\item Ist die FIFO voll, ist \texttt{enq} möglich, wenn zeitgleich ein \texttt{deq} stattfindet. \texttt{first} liefert in dem Falle den alten Wert (vor \texttt{enq}).
				\end{itemize}
			% end
			
			\paragraph{Bypass FIFO}
				\begin{itemize}
					\item Ist die FIFO leer, ist \texttt{deq} möglich, wenn zeitgleich in \texttt{enq} stattfindet. \texttt{first} liefert in dem Falle den neuen Wert (nach \texttt{enq}).
					\item Ist die FIFO voll, ist kein \texttt{enq} möglich.
				\end{itemize}
			% end
		% end
		
		\subsection{Concurrent Registers}
			\begin{itemize}
				\item Halten eine Historie von Werten innerhalb eines Taktes
				\item Hierdurch ist mehrmaliges Schreiben und Lesen nach Schreiben möglich.
				\item Präzedenzrelation
					\begin{lstlisting}
Reg#(Bool) ports[N] <- mkCReg(N, False);

ports[0]._read < ports[0]._write <
ports[1]._read < ports[1]._write <
(* $ \cdots $ *)
ports[N - 1]._read < ports[N - 1].write
					\end{lstlisting}
			\end{itemize}
		% end
		
		\subsection{Attribute}
			Attribute werden in Runden Klammern mit Sternchen angegeben. Beispiel: \texttt{(* <attribute> *)}
			
			\subsubsection{Scheduling Attribute}
				\begin{description}
					\item[\texttt{descending\_urgency}] Reihenfolge/Priorität der Berechnung der \texttt{WILL\_FIRE} Bedingungen (Dringlichkeit).
					\item[\texttt{execution\_order}] Reihenfolge/Priorität der Ausführungsreihenfolge des Regelkörpers (Frühzeitigkeit).
					\item[\texttt{preempts}] Erlaubt einer gefeuerten Regel, das Ausführen anderer Regeln zu unterdrücken (auch wenn kein Konflikt vorliegt).
					\item[\texttt{mutually\_exclusive}] Zusicherung an dem Compiler, dass zwei Regelbedingungen \textit{niemals zeitgleich} wahr sind (bspw. bei One-Hot-Kodierungen).
				\end{description}
			% end
			
			\subsubsection{Synthetisierung Attribute}
				\begin{description}
					\item[\texttt{synthesize}] Verhindert das Inlining eines Moduls. Kann nur angewandt werden auf Module, welche ausschließlich Schnittstellen mit Bits, Skalaren und Bit-Vektoren hat (da Verilog weniger mächtig ist als Bluespec).
				\end{description}
			% end
		% end
	% end
% end

\chapter{Hardware-Entwurfstechniken}
	\section{Entwurfsebenen (Abstraktionsgrad)}
		\begin{enumerate}
			\item Verhaltensebene
				\begin{itemize}
					\item Was soll passieren? Die Realisierung bleibt offen.
				\end{itemize}
			\item Systemebene
				\begin{itemize}
					\item Grobe Aufteilung von Struktur, Zeit, Daten und Kommunikation (CPU, FPGA, DRAM, \dots)
				\end{itemize}
			\item Register-Transfer-Ebene (RTL, Register-Transfer-Logic)
				\begin{itemize}
					\item Synchron, getaktet
				\end{itemize}
			\item Logik- oder Gatterebene
				\begin{itemize}
					\item Netze aus Gattern, Flip-Flops, \dots
				\end{itemize}
			\item Transistorebene
				\begin{itemize}
					\item Elektrischer Schaltplan
				\end{itemize}
			\item Layoutebene
				\begin{itemize}
					\item Maßstabsgetreue geometrische Anordnung des Chips mit verschiedenen Schichten (3D)
				\end{itemize}
		\end{enumerate}
	% end
	
	\section{Ablauf der Synthese}
		Siehe \ref{fig:synthesis}.

		\begin{figure}[ht]
			\centering
			\begin{tikzpicture}[every node/.style = { draw, align = center, minimum width = 5cm }, repr/.style = { rounded corners }]
				\node [repr] (reprRtl) {RTL-Modell};
				\node [below = 1 of reprRtl] (stepHdlc) {HDL-Compiler};
				\node [below = 1 of stepHdlc, repr] (reprSec) {unoptimierte \\ Zwischendarstellung};
				\node [below = 1.5 of reprSec] (stepLogic) {Logikoptimierung};
				\node [below = 1 of stepLogic] (stepTech) {Technology-Mapping};
				\node [below = 1.5 of stepTech, repr] (reprGate) {optimiertes Gattermodell};
				\node [right of = stepLogic, node distance = 7cm] (design) {Design-Constraints};
				\node [right of = stepTech, node distance = 7cm] (tech) {Technologie-Bibliothek};
				
				\coordinate [below = 1 of reprGate] (after);
				\coordinate [above left = 1 of stepLogic] (topLeft);
				\coordinate [above right = 1 of stepLogic] (topRight);
				\coordinate [below right = 1 of stepTech] (bottomRight);
				\coordinate [below left = 1 of stepTech] (bottomLeft);
				
				\draw [-{Latex[length=3mm]}] (reprRtl) -- (stepHdlc);
				\draw [-{Latex[length=3mm]}] (stepHdlc) -- (reprSec);
				\draw [-{Latex[length=3mm]}] (reprSec) -- (stepLogic);
				\draw [-{Latex[length=3mm]}] (stepLogic) -- (stepTech);
				\draw [-{Latex[length=3mm]}] (stepTech) -- (reprGate);
				\draw [-{Latex[length=3mm]}] (reprGate) -- (after);
				\draw [-{Latex[length=3mm]}] (design) -- (stepLogic);
				\draw [-{Latex[length=3mm]}] (design) -- (stepTech);
				\draw [-{Latex[length=3mm]}] (tech) -- (stepTech);
				
				\draw (topLeft) -- (topRight);
				\draw (topRight) -- (bottomRight);
				\draw (bottomRight) -- (bottomLeft);
				\draw (bottomLeft) -- (topLeft);
			\end{tikzpicture}
			\caption{Ablauf der Synthese}
			\label{fig:synthesis}
		\end{figure}
	% end
	
	\section{Design Constraints (Einschränkungen)}
		\begin{itemize}
			\item Zeit
				\begin{itemize}
					\item Timing-Analyse
					\item Geschätzt nach Synthese (\textit{ohne} Verdrahtungsverzögerung)
					\item Exakt nach Platzieren und Verdrahten
					\item $ \rightarrow $ Layoutebene
				\end{itemize}
			\item Fläche
				\begin{itemize}
					\item Geschätzt nach Synthese (\textit{ohne} Verdrahtungsfläche)
					\item Exakt nach Platzieren und Verdrahten
				\end{itemize}
			\item Elektischer Leistungsaufnahme
				\begin{itemize}
					\item Simulation auf Layout-Ebene
					\item Bestimmung von Umschaltfrequenzen (\textit{toggle frequencies}) von Signalen
				\end{itemize}
		\end{itemize}
	% end
	
	\section{Verifikation}
		\subsection{Prä-Synthese-Simulation}
			\begin{itemize}
				\item Testen der Implementierung in der HDL (Simulation)
			\end{itemize}
		% end
		
		\subsection{Post-Synthese-Simulation}
			\begin{itemize}
				\item Gleiche Testdaten wie bei der Prä-Synthese-Simulation
				\item Vergleichen der Ergebnisse
				\item Vergleiche sind Aufgrund des Zeitverhaltens nicht trivial
					\begin{itemize}
						\item Unterschiedlich viele Ergebnisse
						\item Verschiedene Werte
						\item Unterschiedliche Zeiten
						\item Interpretation nötig
					\end{itemize}
			\end{itemize}
		% end
		
		\subsection{Post-Layout-Simulation}
			\begin{itemize}
				\item Gleiche Testdaten wie bei der Prä-/Post-Synthese-Simulation
				\item Gleiches Vorgehen wie bei der Post-Synthese-Simulation
				\item Schließt Gatter- und Leitungsverzögerungen mit ein
				\item Kann auch Widerstände, Kapazitäten, Induktivitäten, \dots enthalten
			\end{itemize}
		% end
	% end
% end

\chapter{Rekonfigurierbare System-on-Chips (SoCs)}
	\section{ARM Cortes-A9 Prozessorkern}
		\begin{itemize}
			\item Superskalar Out-of-Order
			\item Holt zwei Instruktionen je Takt
			\item Kann je Takt bis zu vier Instruktionen ausführen
				\begin{itemize}
					\item ALU/MUL
					\item ALU
					\item FPU/SIMD
					\item Load-Store
				\end{itemize}
			\item Mehr ILP (Instruction Level Parallelism) durch:
				\begin{itemize}
					\item Umbenennung von Registern (\textit{renaming})
					\item Dynamisches Vorziehen von unabhängigen Instruktionen (\textit{out-of-order execution})
				\end{itemize}
		\end{itemize}
	% end
	
	\section{SIMD Rechnungen mit NEON}
		\begin{itemize}
			\item \textit{Single Instruction Multiple Data} (die gleiche Operation wird auf mehreren Daten ausgeführt)
			\item Wird auch als normale FPU (auf skalaren Daten) genutzt
		\end{itemize}
	% end
	
	\section{Speichersystem}
		\begin{itemize}
			\item Prefetching
				\begin{itemize}
					\item Beobachte Speicherzugriffe des Prozessors
					\item Holt die Daten schon mal \enquote{vorweg} in L1-D\$
					\item Beobachtet dabei bis zu acht unabhängige Datenströme
				\end{itemize}
			\item Cache-Kohörenz
				\begin{itemize}
					\item Problem
						\begin{itemize}
							\item Prozessorkerne haben eigenen L1-Cache
							\item Greifen aber auf den gemeinsamen Hauptspeicher und L2-Cache zu
						\end{itemize}
					\item Prozessorkerne müssen austauschen, wo welcher Wert liegt
					\item Aufgabe wird von Snoop-Control-Unit übernommen
						\begin{itemize}
							\item Überwacht Speicherzugriffe
							\item Gibt aktuellen Wert weiter
							\item Protokoll: MESI (Modified/Exclusive/Shared/Invalid)
						\end{itemize}
				\end{itemize}
			\item On-Chip-Memory
				\begin{itemize}
					\item 256 KB SRAM direkt auf dem Chip
					\item Nicht sonderlich schnell (1.400 MB/s) (Vergleich: Externes 32b DDR3-SDRAM 1066 ließ 3.700 MB/s)
					\item Aber sehr geringe Latenz (32 bis 34 Taktzyklen) (Vergleich: Externer 32b DDR3-SDRAM hat 37 bis 200 Zyklen)
				\end{itemize}
		\end{itemize}
	% end

	
	\section{Programmierbare Logik}
		\subsection{Konfigurierbarer Logikblock}
			\begin{itemize}
				\item Je zwei Slices mit je 4 LUTs und 8 Flip-Flops
				\item Verbunden mit der Switch Matrix
			\end{itemize}
		% end
	
		\subsection{Hard Blocks}
			\begin{itemize}
				\item Integrierter Speicher (BRAM)
					\begin{itemize}
						\item Je 36 Kb Speicher
						\item Wortbreite konfigurierbar
						\item Alle Blöcke parallel zugreifbar
					\end{itemize}
				\item Integrierter Multiplizierer (DSP)
			\end{itemize}
		% end
	% end
	
	\section{Hard Core vs. Soft Core}
		\begin{description}
			\item[Hard Core] In ASIC erstellter Prozessor.
			\item[Soft Core] Auf einem FPGA realisierter Prozessor.
		\end{description}
		
		\paragraph{Vergleich Hard/Soft Core}
			\begin{itemize}
				\item Soft Cores in rekonfigurierbarer Logik
					\begin{itemize}
						\item brauchen deutlich mehr Chip-Fläche als Hard Cores
						\item laufen wesentlich langsamer.
					\end{itemize}
				\item Einsatz von Soft Cores i.d.R. nicht mehr effizient, Trend geht zu Kombination von
					\begin{itemize}
						\item Hard Core CPUs
						\item Rekonfigurierbarer Logik
					\end{itemize}
			\end{itemize}
		% end
	% end
	
	\section{Schnittstelle PS $ \leftrightarrow $ PL}
		\paragraph{General Purpose Interface (GP)}
			\begin{itemize}
				\item Generelles AXI-Interface für (fast) alle Anforderungen
				\item Direkter Zugriff auf den Speicher
				\item Anschluss für Peripherie
				\item Hat 2 Master und 2 Slave-Ports (bidirektional)
				\item Cache-inkohärent zu der CPU
			\end{itemize}
		% end
		
		\paragraph{High Performance Interface (HP)}
			\begin{itemize}
				\item Hochperformantes AXI-Interface für schnelle Datenübertragung
				\item Direkter Zugriff auf den Speicher
				\item Kein Peripherie-Anschluss, daher geringe Belastung
				\item Cache-inkohärent zu der CPU
				\item Zugriff immer direkt auf den RAM, belastet nicht den L2-Cache der CPU
			\end{itemize}
		% end
		
		\paragraph{Amber Coherence Protocol Interface (ACP)}
			\begin{itemize}
				\item Zugriff auf die SCU (Snoop-Control-Unit)
				\item Daher Beibehaltung der Cache-Kohärenz
				\item Liefert schnell Daten, solange diese im Cache liegen (sonst sehr langsam)
				\item Teilt sich L2-Cache mit der CPU
			\end{itemize}
		% end
	% end
	
	\section{AXI4}
		\subsection{Protokollfamilie}
			\paragraph{AXI4}
				\begin{itemize}
					\item Mächtigste Implementierung
					\item Benötigt am meisten Chipfläche
					\item Unterstützt memory-mapped I/O
					\item Erlaubt effiziente Übertragung von Datengruppen (\textit{burst transfer})
				\end{itemize}
			% end
			
			\paragraph{AXI3-Lite}
				\begin{itemize}
					\item Einfacher als AXI4
					\item Benötigt weniger Chipfläche als AXI4
					\item Unterstützt memory-mapped I/O
					\item Erlaubt keine Burst-Transfers, sonder nur einzelne Daten
				\end{itemize}
			% end
			
			\paragraph{AXI4-Stream}
				\begin{itemize}
					\item Spezialisierte Realisierung (z.B. für Sensoren)
					\item Überträgt nur reine Datenströme (keine Adressen und somit kein memory-mapped I/O)
					\item Unbegrenzt lange Bursts
					\item Unidirektionale Übertragung von Master zu Slave
				\end{itemize}
			% end
		% end
		
		\subsection{Grundkonzepte AXI4}
			\begin{itemize}
				\item Master löst Übertragung aus, Slave reagiert auf die Übertragung
				\item Slave liefer Daten an Master bei Lesezugriff
				\item Slave nimmt Daten vom Master entgegen bei Schreibzugriff
				\item Slave liefert Status an Master (über Lesekanal oder extra Rückkanal für Antworten bei Schreibzugriffen)
			\end{itemize}
			
			
			Bei einer Burst-Übertragung setzt der Master nur die Start-Adresse, der Slave muss die Folgeadressen selbst bestimmen (bspw. durch inkrementieren).
			
			Die maximale Länge eines Bursts ist bei AXI4 256 Datensätze (genannt \textit{beats}). Ein Beat kann 1 bis 128 Bytes umfassen.
		% end
		
		\subsection{Grundlagen der Signalisierung}
			\begin{itemize}
				\item Handshake zwischen Quelle und Ziel von Daten, ausgewertet bei \texttt{@posedge}
					\begin{itemize}
						\item Quelle setzt \texttt{VALID}, wenn gültige Daten anliegen
						\item Ziel setzt \texttt{READY}, wenn Daten übernommen werden können
						\item Sind \texttt{VALID} und \texttt{READY} zeitgleich gesetzt, wurden die Daten übernommen $ \rightarrow $ \texttt{READY} fällt.
					\end{itemize}
				\item Es existieren die folgenden Signale (Kanal $ \rightarrow $ Handshake Paar):
					\begin{description}[leftmargin = 5cm]
						\item[Write Address Channel] \texttt{AWVALID}, \texttt{AWREADY}
						\item[Write Data Channel] \texttt{WVALID}, \texttt{WREADY}
						\item[Write Response Channel] \texttt{BVALID}, \texttt{BREADY}
						\item[Read Address Channel] \texttt{ARVALID}, \texttt{ARREADY}
						\item[Read Data Channel] \texttt{RVALID}, \texttt{RREADY}
					\end{description}
			\end{itemize}
		% end
	% end
	
	\section{IP-Blöcke}
		\begin{itemize}
			\item \enquote{Intellectual Property}
			\item Vordefinierte Hardware-Funktionen
			\item Meist stark konfigurierbar
		\end{itemize}
	% end
% end

\chapter{Entwurfsverfahren und Werkzeuge für Hardware-Beschleuniger}
	\section{Hardware-Klassifikation}
		\subsection{Hardware-Arten}
			\begin{description}
				\item[ASIC] Application Specific Integrated Circuit
				\item[microController] Instruction Set Architecture (ISA), limitierter Einsatzbereich
				\item[System-on-Chip] Kleinskalierte Architekruren
				\item[Low-Power-CPU] Desktop-CPUs mit Fokus auf Energieeffizienz
				\item[Multi-Core CPU/SoC] Desktop und Server CPUs, SoCs
				\item[GPGPU] General Purpose GPU
				\item[Many-Core CPU] Massivparallele CPUs
				\item[DSP] Digital Dignal Processor, Massivparallele Arithmetikeinheiten
				\item[FPGA] Field Programmable Logic Gate Array
			\end{description}
		% end
		
		\subsection{Klassifikation}
			\begin{description}
				\item[Commodity ISAs] microController, LPCPUs, Multi-Core CPUs, SoCs
					\begin{itemize}
						\item Standardisiertes Instruction Set
						\item Starke Unterstützung durch Tools (Compiler, Debugger, \dots)
						\item Programmierbar in Standardsprachen (C, C++, \dots)
					\end{itemize}
				\item[Specialized ISAs] GPGPUs, DSPs, Many-Cores
					\begin{itemize}
						\item Nicht-Standardisiertes Instruction Set
						\item Limitierter Unterstützung durch Tools (Herstellertools)
						\item Programmierbar in spezialisierten Sprachen (OpenMP, CUDA, OpenCP, \dots)
					\end{itemize}
				\item[Reconfigurable Technology] PLDs, FPGAs
					\begin{itemize}
						\item Eigenes Design, kein Standard
						\item Limitierte Unterstützung durch Tools (Herstellertools)
						\item Setzt HDLs voraus
						\item Limitierte Unterstützung von Standardsprachen und spezialisierten Sprachen (C, C++, OpenCL, \dots)
					\end{itemize}
				\item[ASICs] Applikationsspezifische Hardware (bspw. Bitcoin Mining, Verschlüsselung)
					\begin{itemize}
						\item Eigenes Design, keine Standards
						\item Kein Compilersupport (OS-Integration, APIs, \dots sind selber zu entwickeln)
						\item Setzt HDLs voraus
						\item Nur Programmierbar in Low-Level Sprachen, bzw. Hardware-Schnittstellen
					\end{itemize}
			\end{description}
		% end
		
		\subsection{Hardware-Anforderungen}
			\todo{Slides 17, 18, 19}
		% end
	% end
	
	\section{Moore's Law}
		\paragraph{Aussage}
			\textbf{Die Anzahl der Transistoren pro Fläche verdoppelt sich alle zwei Jahre.}
			
			Hat sich 10 Jahre gehalten!
		% end
		
		\paragraph{Ende von Moore's Law}
			\begin{itemize}
				\item 2014 wurden 14nm nur knapp erreicht
				\item 2017 wurden 10nm nicht erreicht
				\item 2020 sollen 5nm erreicht werden, sehr fraglich
			\end{itemize}
		% end
	% end
	
	\section{Patterson's Walls}
		\paragraph{Aussage}
			\textbf{Power Wall + Memory Wall + ILP Wall = Brick Wall}
			
			Wenn man zwei der Mauern durchbricht, scheitert man an der dritten.
			
			Des hat sich für alle bisherigen Computersysteme bewahrheitet (Energieeffizienz vs. Performanz, Performanz vs. Speicher, \dots).
		% end
	% end
	
	\section{FPGA Entwicklungsablauf}
		\begin{enumerate}
			\item Auswahl oder Bau des FPGAs
			\item Erstellung oder Anpassung des Basisdesigns
			\item Implementierung der Logik
			\item Simulation des Designs
			\item Platzieren und Verdrahten
			\item Testen auf echter Hardware
		\end{enumerate}
		
		Dabei werden die Schritte 1 und 2 als \enquote{Hardware Bring-Up}, die Schritte 3 und 4 als \enquote{Logikdesign} und die Schritte 5 und 6 als \enquote{Hardware Synthese} bezeichnet.
	% end
% end

\chapter{TaPaSCo}
    \todo{tapasco}
% end

\chapter{Abkürzungen}
    \begin{description}
		\item[ACP] Amber Coherence Protocol
		\item[ALU] Arithmetic Logical Unit
		\item[AMBA] Advanced Microcontroller Bus Architecture
		\item[API] Application Programming Interface
		\item[APU] Application Processing Unit
		\item[ASIC] Applicaton Specific Integrated Circuit
		\item[AXI] Advanced eXtensible Interface
		\item[BRAM] Block RAM
		\item[BSV] Bluespec System Verilog
		\item[CE] Clock Enable
		\item[CF] Conflict Free
		\item[CLB] Configurable Logic Blocks
		\item[CLK] Clock
		\item[COTS] Commercial Off The Shelf
		\item[CPU] Central Processing Unit
		\item[DDR] Double Data Read
		\item[DMA] Direct Memory Access
		\item[DRAM] Direct RAM
		\item[DSP] Digital Signal Processors
		\item[FIFO] First In - First Out Queue
		\item[FPGA] Field Programmable Gate Array
		\item[FPU] Floating Point Unit
		\item[GALS] Global Asynchronous, Locally Synchronous
		\item[GPGPU] General Purpose GPU
		\item[GPIO] General Purpose IO
		\item[GPU] Graphics Processing Unit
		\item[GUI] Graphical User Interface
		\item[HDL] Hardware Description Language
		\item[HLS] High-Level Synthesis
		\item[HP] High Performance Ports
		\item[HPC] High Performance Computing
		\item[HW] Hardware
		\item[IC] Integrated Circuit
		\item[IDE] Integrated Development Environment
		\item[ILP] Instruction Level Parallelism
		\item[IO] Input/Output
		\item[IP] Intellectual Property
		\item[IPC] Instructions Per Cycle
		\item[ISA] Instruction Set Architecture
		\item[L1D] Level 1 Data Cache
		\item[LCA] Lumped Circuit Abstraction
		\item[LPCPU] Low Power CPU
		\item[LUT] Look Up Table
		\item[MESI] Modified/Exclusive/Shared/Invalid Protocol
		\item[MIPS] Microprocessor without Interlocked Pipeline Staged
		\item[MMU] Memory Management Unit
		\item[MUX] Multiplexer
		\item[NRC] Non-Recurring Costs
		\item[OCM] On-Chip Memory
		\item[OoO] Out of Order (Execution)
		\item[OS] Operating System
		\item[PC] Program Counter
		\item[PCB] Process Control Block
		\item[PCI] Peripheral Component Interconnect
		\item[PCIe] PCI Express
		\item[PE] Processing Elements
		\item[PL] Programmable Logic
		\item[PLD] Programmable Logic Device
		\item[PS] Processing System
		\item[QFN] Quad Flat No Leads Package
		\item[QFP] Quad Flat Package
		\item[RAM] Random Access Memory
		\item[RAW] Read after Write
		\item[RDY] Ready
		\item[RISC] Reduced Instruction Set Computer
		\item[RLDRAM] Reduced Latency DRAM
		\item[RTL] Register-Transfer-Ebene
		\item[SAMD] Shared Address Multiple Data
		\item[SCU] Snoop Control Unit
		\item[SD] Secure Digital Memory Card
		\item[SIMD] Single Instruction Multiple Data
		\item[SoC] System on Chip
		\item[SOP] Small Outline Package
		\item[SSE] Streaming SIMD Extension
		\item[SSH] Secure Shell
		\item[SW] Software
		\item[TaPaSCo] Task Parallel System Composer
		\item[TLB] Translation Lookaside Buffer
		\item[TTL] Transistor-Transistor Logik
		\item[TTM] Time-to-Market
		\item[USB] Universal Serial Bus
		\item[VGA] Video Graphics Array
		\item[VHDL] Very High Speed Integrated Circuit Hardware Description Language
    \end{description}
% end
