\chapter{Einführung} % 1.5
    \todo{Content}

    \section{Beispiele} % 1.6, 1.7, 1.8, 1.10, 1.11
        \todo{Content}
    % end

    \section{Fragestellungen} % 1.12
        \todo{Content}
    % end

    \section{Allgemeine Formulierung eines Optimierungsproblems} % 1.13, 1.14
        \todo{Content}
    % end

    \section{Statische vs. Dynamische Optimierung} % 1.15, 1.16, 1.17, 1.18
        \todo{Content}
    % end

    \section{Klassifizierung von Optimierungsverfahren} % 1.23
        \todo{Content}
    % end

    \section{Typische Struktur} % 1.24
        \todo{Content}
    % end
% end

\chapter{Gradientenbasierte Optimierung ohne Beschränkungen} % 2.2
    \todo{Content}

    \section{Charakterisierung der Lösung} % 2.2
        \todo{Content}

        \subsection{Eindimensionale Optimierung} % 2.3, 2.4, 2.5
            \todo{Content}
        % end

        \subsection{Mehrdimensionale Optimierung} % 2.6, 2.10
            \todo{Content}

            \paragraph{Beispiel} % 2.7, 2.8, 2.9
                \todo{Content}
	        % end
        % end
    % end

    \section{Numerische Gradientenbasierte Verfahren} % 2.11
        \todo{Content}

        \subsection{Ausgangssituation} % 2.13
            \todo{Content}

            \subsubsection{Struktur gradientenbasierter Verfahren} % 2.14
                \todo{Content}
	        % end

            \subsubsection{Abstiegsrichtung} % 2.15
                \todo{Content}
	        % end

            \subsubsection{Algorithmische Struktur} % 2.16
                \todo{Content}
	        % end
        % end

        \subsection{Methode des steilsten Abstiegs (Steepest Descent)} % 2.17
            \todo{Content}
        % end

        \subsection{Methode der konjugierten Gradienten (Conjugate Gradient)} % 2.18, 2.19, 2.20
            \todo{Content}
        % end

        \subsection{Newton-Verfahren} % 2.21, 2.22, 2.23
            \todo{Content}

            \subsubsection{Verfügbarkeit der zweiten Ableitungen} % 2.24
                \todo{Content}
	        % end
        % end

        \subsection{Quasi-Newton-Verfahren} % 2.25, 2.26, 2.27
            \todo{Content}

            \subsubsection{BFGS-Aktualisierung} % 2.28
                \todo{Content}
	        % end
        % end

        \subsection{Vergleich der Verfahren} % 2.29, 2.30, 2.31, 2.32, 2.33, 2.34, 2.35, 2.36, 2.37, 2.38
            \todo{Content}
        % end
    % end

    \section{Schrittweitenbestimmung, Liniensuche} % 2.39, 2.40, 2.47
        \todo{Content}

        \subsection{Charakterisierung der Lösung} % 2.41
            \todo{Content}
        % end

        \subsection{Übersicht und Anforderungen} % 2.42, 2.43
            \todo{Content}
        % end

        \subsection{Näherungsweise Liniensuche} % 2.44, 2.45, 2.46
            \todo{Content}
        % end
    % end

    \section{Vertrauensbereichsverfahren (Trust Region Methods)} % 2.48, 2.49, 2.50
        \todo{Content}
    % end

    \section{Konvergenzraten} % 2.51, 2.52, 2.53, 2.54
        \todo{Content}

        \subsection{Gradientenbasierte Verfahren} % 2.55
            \todo{Content}
        % end
    % end
% end

\chapter{Gradientenfreie Optimierung ohne Beschränkungen} % 3.1, 3.2
    \todo{Content}

    \section{Einleitung} % N/A
    	\todo{Content}

    	\subsection{Historische Entwicklung} % 3.3
        	\todo{Content}
	    % end

        \subsection{Problemstellung} % 3.4
            \todo{Content}
        % end

        \subsection{Simulationsbasierte Optimierung} % 3.5, 3.6, 3.7, 3.8
            \todo{Content}
        % end

        \subsection{Black-Box Optimierung} % 3.9
            \todo{Content}
        % end
    % end

    \section{Metaheuristiken} % 3.10
        \todo{Content}

        \subsection{Evolutionäre Algorithmen (EA)} % 3.11
            \todo{Content}
        % end

        \subsection{Genetische Algorithmen (GA)} % 3.12
            \todo{Content}

            \paragraph{Beispiel} % 3.13, 3.14
                \todo{Content}
		    % end
        % end

        \subsection{Weitere Metaheursitiken} % 3.15
            \todo{Content}
        % end
    % end

    \section{Deterministische Sampling Verfahren (Mustersuchverfahren)} % 3.16, 3.17
        \todo{Content}

        \subsection{Nelder-Mead Simplexverfahren} % 3.18, 3.26, 3.27
            \todo{Content}

            \subsubsection{Iterationsphase} % 3.19, 3.20
                \todo{Content}
	        % end

            \subsubsection{Algorithmus} % 3.21, 3.22, 3.23
                \todo{Content}
	        % end
        % end

        \subsection{Multidirektionales Suchverfahren (MDS)} % 3.28
            \todo{Content}
        % end

        \subsection{Asynchronous Parallel Pattern Search (APPS)} % 3.29, 3.30, 3.31
            \todo{Content}

            \paragraph{Beispiel} % 3.32, 3.33, 3.34
                \todo{Content}
	        % end
        % end

        \subsection{Implizites Filtern} % 3.35, 3.36, 3.37, 3.38
            \todo{Content}
        % end
    % end

    \section{Optimierung mit Ersatzfunktionen} % 3.39, 3.40, 3.41
        \todo{Content}

        \subsection{Approximationsmethoden} % 3.42
            \todo{Content}

            \subsubsection{Response Surface Methoden (RSM)} % 3.43
                \todo{Content}
	        % end

            \subsubsection{Radial Basis FUnctions (RBF)} % 3.44
                \todo{Content}
	        % end

            \subsubsection{Design and Analysis of Computer Experiments (DACE)} % 3.45
                \todo{Content}
	        % end
        % end

        \subsection{Auswahl der Stützstellen/Datenbasis/Anfangswerte} % 3.46
            \todo{Content}

            \subsubsection{Design of Experiments (DoE)} % 3.47
                \todo{Content}
	        % end
        % end

        \subsection{Minimierung der Ersatzfunktion} % 3.48
            \todo{Content}

            \subsubsection{Optimierung am Ersatzproblem} % 3.49
                \todo{Content}

                \paragraph{Strawman} % 3.48
                    \todo{Content}
		        % end

                \paragraph{Shoemaker} % 3.48
                    \todo{Content}
		        % end

                \paragraph{DACE-basierte, sequentielle Update-Strategien} % 3.50
                    \todo{Content}
		        % end
	        % end
        % end

        \subsection{Diskussion} % 3.54
            \todo{Content}
        % end
    % end

    \section{Vergleich der Verfahren} % 3.55
        \todo{Content}

        \subsection{Design eines Magnetlagers} % 3.56, 3.57, 3.58, 3.59, 3.60, 3.61
            \todo{Content}
        % end

        \subsection{Laufoptimierung eines Humanoidroboters} % 3.62, 3.63
            \todo{Content}
        % end
    % end

    \section{Diskussion} % 3.65, 3.66
        \todo{Content}
    % end
% end

\chapter{Gradientenbasierte Optimierung mit Beschränkungen} % 4.1, 4.2, 4.3, 4.4, 4.5
    \todo{Content}

    \section{Charakterisierung der Lösung} % 4.6
        \todo{Content}

        \subsection{Motivation} % 4.7, 4.8
            \todo{Content}
        % end

        \subsection{Lagrange-Funktion} % 4.9
            \todo{Content}
        % end

        \subsection{Notwendige Optimalitätsbedingungen 1. Ordnung (Karush-Kuhn-Tucker, KKT)} % 4.10
            \todo{Content}
        % end

        \subsection{Notwendige Optimalitätsbedingungen 2. Ordnung} % 4.12
            \todo{Content}
        % end

        \subsection{Beispiel} % 4.11, 4.13
            \todo{Content}
        % end
    % end

    \section{Einfache Schranken} % 4.14, 4.15, 4.16, 4.17
        \todo{Content}
    % end

    \section{Straffunktionsverfahren} % 4.18, 4.19
        \todo{Content}

        \subsection{Äußere Straffunktionsverfahren} % 4.20, 4.21
            \todo{Content}

            \paragraph{Beispiel} % 4.22, 4.23
                \todo{Content}
	        % end
        % end

        \subsection{Innere Straffunktionsverfahren} % 4.24, 4.25
            \todo{Content}

            \paragraph{Beispiel} % 4.26
                \todo{Content}
	        % end
        % end

        \subsection{Exakte Straffunktionen} % 4.27
            \todo{Content}

            \paragraph{Beispiel 1} % 4.28
                \todo{Content}
	        % end

            \paragraph{Beispiel 2} % 4.29
                \todo{Content}
	        % end
        % end

        \subsection{Erweiterte Lagrange-Funktion} % 4.30
            \todo{Content}

            \paragraph{Beispiel 1} % 4.31, 4.32
                \todo{Content}
	        % end

            \paragraph{Beispiel 2} % 4.33
                \todo{Content}
	        % end

            \subsubsection{Eigenschaften} % 4.34, 4.35
                \todo{Content}
	        % end
        % end
    % end

    \section{Elimination von Beschränkungen} % 4.36, 4.37, 4.41
        \todo{Content}

        \paragraph{Beispiel 1} % 4.38
            \todo{Content}
        % end

        \paragraph{Beispiel 2} % 4.39
            \todo{Content}
        % end

        \paragraph{Beispiel 3} % 4.40
            \todo{Content}
        % end
    % end

    \section{Verfahren der Sequentiellen Quadratischen Optimierung (SQP)} % 4.42
        \todo{Content}

        \subsection{Einleitung} % 4.43, 4.44
            \todo{Content}
        % end

        \subsection{Bestimmung der Suchrichtung} % 4.45, 4.46, 4.47
            \todo{Content}

            \subsubsection{Quadratisches Problem (QP)} % 4.48
                \todo{Content}
	        % end
        % end

        \subsection{Bestimmung der Schrittweite} % 4.50
            \todo{Content}
        % end

        \subsection{Approximation der Lagrange-Multiplikatoren} % 4.52
            \todo{Content}
        % end

        \subsection{Terminierungskriterien} % 4.54, 4.55
            \todo{Content}
        % end

        \subsection{Approximation der Hesse-Matrix} % 4.58, 4.59
            \todo{Content}

            \subsubsection{Naiver Ansatz} % 4.60
                \todo{Content}
	        % end

            \subsubsection{Reduzierte Hesse-Matrix} % 4.61, 4.63, 4.64, 4.65, 4.66, 4.67, 4.68
                \todo{Content}

                \paragraph{Beispiel} % 4.62
                    \todo{Content}
		        % end
	        % end

            \subsubsection{Approximation der reduzierten Hesse-Matrix} % 4.69
                \todo{Content}
	        % end
        % end

        \subsection{SQP-Verfahren} % 4.52, 4.53, 4.57, 4.75
            \todo{Content}
        % end

        \subsection{Bemerkungen} % 4.49, 4.70, 4.75, 4.76, 4.77, 4.78, 4.79
            \todo{Content}
        % end

        \subsection{Beispiele} % N/A
            \todo{Content}

            \subsubsection{Optimale Steuerung eines 6-gelenkigen Industrieroboters} % 4.71, 4.72
                \todo{Content}
	        % end

            \subsubsection{PKW-Fahrt} % 4.73, 4.74
                \todo{Content}
	        % end
        % end
    % end
% end

\chapter{Berechnung von Ableitungen} % 5.1, 5.2, 5.3, 5.4
    \todo{Content}

    \section{Finite-Differenzen-Approximation (numerische Differentiation)} % 5.5
        \todo{Content}

        \subsection{Vorwärtsdifferenzen-Approximation} % 5.6, 5.13
            \todo{Content}

            \subsubsection{Fehler} % 5.8, 5.12
                \todo{Content}

                \paragraph{Approximationsfehler} % 5.9
                    \todo{Content}
		        % end

                \paragraph{Funktionsgenauigkeit} % 5.10
                    \todo{Content}
		        % end

                \paragraph{Rundungsfehler} % 5.11
                    \todo{Content}
    		    % end
	        % end

            \subsubsection{Wahl der Schrittweite} % 5.14, 5.15, 5.16, 5.17
                \todo{Content}
	        % end
        % end

        \subsection{Zentrale-Differenzen-Approximation} % 5.18, 5.19, 5.21, 5.22
            \todo{Content}
        % end
    % end

    \section{Numerische Differentiation von Simulationsmodellen} % 5.23, 5.24, 5.25
        \todo{Content}

        \subsection{Ableitung von ODE-Simulationsmodellen} % 5.27, 5.28
            \todo{Content}
        % end

        \subsection{Externe numerische Differentiation} % N/A
            \todo{Content}

            \subsubsection{Naiver Ansatz} % 5.29, 5.30, 5.31
                \todo{Content}
	        % end

            \subsubsection{Gekoppelte Vorwärtsdifferenzen-Approximation} % 5.32
                \todo{Content}
	        % end
        % end

        \subsection{Interne numerische Differentiation} % 5.33, 5.34
            \todo{Content}
        % end
    % end

    \section{Symbolische Differentiation} % 5.35, 5.36
        \todo{Content}
    % end

    \section{Automatisches Differenzieren} % 5.37, 5.38, 5.39, 5.40
        \todo{Content}

        \paragraph{Beispiel} % 5.41, 5.42, 5.43
            \todo{Content}
        % end
    % end
% end

\chapter{Minimierung von Abweichungen} % 6.1, 6.2, 6.3
    \todo{Content}

    \section{Parameterschätzung bei ODE-Systemen} % 6.5, 6.7
        \todo{Content}
    % end

    \section{Gütekriterien zur Minimierung von Abweichungen} % 6.8, 6.9
        \todo{Content}
    % end

    \section{Lineare Ausgleichsrechnung} % 6.10, 6.11
        \todo{Content}
    % end

    \section{Optimalitätsbedingungen und Spezialverfahren} % 6.12, 6.13, 6.17
        \todo{Content}

        \subsection{Quasi-Newton} % 6.14
            \todo{Content}
        % end

        \subsection{Gauß-Newton Verfahren} % 6.15
            \todo{Content}
        % end

        \subsection{Levenberg-Marquardt Verfahren} % 6.16
            \todo{Content}
        % end

        \subsection{SQP-Verfahren} % 6.18
            \todo{Content}
        % end
    % end

    \section{Normalen-Gleichungen} % 6.22, 6.23
        \todo{Content}

        \paragraph{Beispiel} % 6.24, 6.25, 6.26, 6.27, 6.28
            \todo{Content}
        % end
    % end

    \section{Interpretation von Berechnungsergebnissen} % 6.29
        \todo{Content}

        \subsection{Mögliche Ursachen für Schwierigkeiten} % 6.30, 6.31
            \todo{Content}
        % end
    % end

    \section{Die Varianz-Kovarianz-Matrix} % 6.32, 6.33, 6.34
        \todo{Content}
    % end

    \section{Optimale Experimentgestaltung} % 6.35, 6.36, 6.37
        \todo{Content}
    % end

    \section{Beispiele} % 6.39
        \todo{Content}

        \subsection{Parameterabhängige Gesamtfahrzeugdynamik} % 6.39, 6.40
            \todo{Content}

            \subsubsection{Simulierte Messwerte} % 6.41, 6.42, 6.43
                \todo{Content}
	        % end

            \subsubsection{Echte Messwerte} % 6.44
                \todo{Content}
	        % end

            \subsubsection{Vergleich der Verfahren} % 6.45
                \todo{Content}
	        % end
        % end

        \subsection{Parameterschätzung für "BioBiped"} % 6.47, 6.48, 6.49
            \todo{Content}
        % end
    % end
% end
